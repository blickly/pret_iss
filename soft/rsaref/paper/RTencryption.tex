%
%  $Description: Author guidelines and sample document in LaTeX 2.09$
%
%  $Author: mankit $
%  $Date: 2008/05/20 07:24:15 $
%  $Revision: 1.41 $
%

\documentclass[times, 10pt,twocolumn]{article}
\usepackage{latex8}
\usepackage{times}
\usepackage{amssymb}
\usepackage{amsmath}
\usepackage{stmaryrd}
\usepackage{graphicx}
\usepackage{fancyvrb}
\usepackage{color}
%\documentstyle[times,art10,twocolumn,lat ex8]{article}

%-------------------------------------------------------------------------
% take the % away on next line to produce the final camera-ready version
\pagestyle{empty}

%-------------------------------------------------------------------------
\begin{document}

  \title{Side Channel Elimination on a Precision Timed Architecture}

  \author{Isaac Liu, David McGrogan \\
    Center for Hybrid and Embedded Software Systems, EECS \\
    University of California, Berkeley \\
    Berkeley, CA 94720, USA \\
    \{\tt liuisaac, dpmcgrog\}@eecs.berkeley.edu
  }


\maketitle
\thispagestyle{empty}

\begin{abstract}
Eliminating side channel attacks for encryptions
\end{abstract}

%encryption info
%	explain a little about the applications of encryption and how it works
%- side channel attack info
%	brief summary on side channel attacks, including timing, cache, branch predictor (i can add this in later), and maybe even power
%- Pret
%	Explanation of the architecture, and deadline instruction (you can take some graphics off our paper or the current poster template)
%-source code
%	I'll add in some source code snap shots of how to program with deadlines
%-Results
%	Plots of data and timing information

\section{Outline and Breakdown}
\begin{itemize}
\item abstract, Introduction  - Isaac
\item Related Work - David
\item Encryption and Side Channel Attacks - David
\item PRET  - Isaac 
\item Results  - Isaac
\item Conclusion and future work - David
\end{itemize}

%------------------------------------------------------------------------------
\section{Introduction}

\section{Related work}

\section{Background}
% 1 - 1.5 page
\subsection{Encryption}
The goal of encryption is to make information illegible to anyone without special knowledge, generally expressed as a key.  It is naturally of tremendous worth in the Information Age, especially to entities such as governments and businesses, which have known adversaries.  Encryption algorithms generally use a so-called trapdoor function, which is easy to compute in one direction but difficult (according to current knowledge) to compute in the other direction without an additional piece of information.  These functions are based on operations such as prime factorization or taking discrete logarithms; as years of concerted effort has failed to produce an efficient algorithm to invert them, the cryptographic algorithms founded on them are used with a large amount of confidence in the algorithms' security.  That is, given all details of the encryption algorithm except for the key, an adversary will not be able to read the encoded information with any reasonable amount of computational power.  [image: traditional model of cryptography]
\subsection{Side Channel Attacks}
Traditional attacks on cryptographic algorithms use only the input and output of the algorithm, treating it like a monolithic black box.  In contrast, side channel attacks circumvent the mathematical complexity of reversing encryption by getting additional data about the encryption process.  [image: model with info. leakage]  Depending on the algorithm and its implementation, a wide variety of information leaks may exist on a number of different physical channels.  The attacks which exploit these channels differ greatly in the difficulty of implementation (or the information is easier or harder to get), ranging from simple in the case of timing attacks to requiring physical access in the case of power attacks.
\subsubsection{Timing attacks}
Timing attacks observe variation in the time spent by an encryption algorithm, often with a known input, and use this information to deduce the key.  This timing data are often compared to a duplicate of the encrypting hardware belonging to the attacker over various keys, enabling better feedback.  Vulnerability to this attack depends on the software implementation of the algorithm, but is rather widespread due to the general drive toward fast algorithms.
\subsubsection{Caching attacks}
Caching attacks use a spy thread running concurrently with the encryption program on the target hardware.  The spy thread occupies all lines in the cache, and detects the loading of data from different locations in memory by timing the return of the data; if the encrypting program has evicted the soy thread's data for its own use, the spy thread's load operation will take longer.  For some algorithms, such as AES, this enables information about the key (which has precomputed components) to be obtained.
\subsubsection{Branch Predictor attacks}
Similar to caching attacks, branch predictor attacks involve a spy program running concurrently with the encryption program.  In this case, the spy thread fills all entries in the branch predictor table with a known value and constantly checks up on those values.  By counting the cycles required for a branch, the spy program detects any change in the state of the branch predictor and therefore infers the control flow of the encryption thread, revealing information about the key.
\subsubsection{Power attacks}
Power attacks use the changing power consumption of the processor to infer the activity of the encryption software over time.  Differences in algorithm activity based on the key will be revealed by the fluctuations they create in processor energy use.  Power attacks require measurement of the power intake of the processor, and are thus generally impossible without physical access to the target hardware, but this is no obstacle in cases such as consumer electronics.

\section{Elimination of Side Channel attacks}
We can begin see a pattern in all of the side channel attacks that are mentioned above. The attacker collects information from the underlying hardware, and use it to infer information regarding the encryption algorithm, which can lead to exposure of the secret key. \cite{Kelsey98sidechannel} offers a more in-depth report on side-channel cryptanalysis and provide examples of them. 

Computer architects have made amazing advancements in architecture design to allow for faster processing. The introduction and improvement of branch predictors and caches allow for better speculative execution. Hardware threading mechanisms such as simultaneous multi-threading improve the utilization and throughput of the processor. However, these improvements come at a cost. Because a portion of the speedups are gained from speculative execution, the amount of complexity required to maintain state and recovery is enormous. As a result, programs executed on modern processors have unpredictable and uncontrollable execution times. 

The branch predictor side channel attack and cache side channel attack both attack a single shared resource from the hardware. By writing a spy process to hog up that resource, you can easily monitor another thread's access to the same resource and therefore monitor the activity of the thread. The timing side channel attack attacks the execution time difference of a processor, which is indirectly caused by its unpredictable hardware components

\subsection{PRET}


\section{Implementation}

\section{Results}



\section{Conclusion}

\bibliographystyle{latex8}
\bibliography{RTencryption}

%-------------------------------------------------------------------------

\end{document}
