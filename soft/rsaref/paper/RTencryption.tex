
%
%  $Description: Author guidelines and sample document in LaTeX 2.09$
%
%  $Author: mankit $
%  $Date: 2008/05/20 07:24:15 $
%  $Revision: 1.41 $
%

\documentclass[times, 10pt,twocolumn]{article}
\usepackage{latex8}
\usepackage{times}
\usepackage{amssymb}
\usepackage{amsmath}
\usepackage{stmaryrd}
\usepackage{graphicx}
\usepackage{fancyvrb}
\usepackage{color}
%\documentstyle[times,art10,twocolumn,lat ex8]{article}

%-------------------------------------------------------------------------
% take the % away on next line to produce the final camera-ready version
\pagestyle{empty}

%-------------------------------------------------------------------------
\begin{document}

  \title{Side Channel Elimination on a Precision Timed Architecture}

  \author{Isaac Liu, David McGrogan \\
    Center for Hybrid and Embedded Software Systems, EECS \\
    University of California, Berkeley \\
    Berkeley, CA 94720, USA \\
    \{\tt liuisaac, dpmcgrog\}@eecs.berkeley.edu
  }


\maketitle
\thispagestyle{empty}

\begin{abstract}
Eliminating side channel attacks for encryptions
\end{abstract}

%encryption info
%	explain a little about the applications of encryption and how it works
%- side channel attack info
%	brief summary on side channel attacks, including timing, cache, branch predictor (i can add this in later), and maybe even power
%- Pret
%	Explanation of the architecture, and deadline instruction (you can take some graphics off our paper or the current poster template)
%-source code
%	I'll add in some source code snap shots of how to program with deadlines
%-Results
%	Plots of data and timing information

\section{Outline and Breakdown}
\begin{itemize}
\item abstract, Introduction  - Isaac
\item Related Work - David
\item Encryption and Side Channel Attacks - David
\item PRET  - Isaac 
\item Results  - Isaac
\item Conclusion and future work - David
\end{itemize}

%------------------------------------------------------------------------------
\section{Introduction}

\section{Related work}

\section{Background}
% 1 - 1.5 page
\subsection{Encryption}
The goal of encryption is to make information illegible to anyone without special knowledge, generally expressed as a key.  It is naturally of tremendous worth in the Information Age, especially to entities such as governments and businesses, which have known adversaries.  Encryption algorithms generally use a so-called trapdoor function, which is easy to compute in one direction but difficult (according to current knowledge) to compute in the other direction without an additional piece of information.  These functions are based on operations such as prime factorization or taking discrete logarithms; as years of concerted effort has failed to produce an efficient algorithm to invert them, the cryptographic algorithms founded on them are used with a large amount of confidence in the algorithms' security.
\subsection{Side Channel Attacks}
Traditional attacks on cryptographic algorithms use only the input and output of the algorithm, treating it like a monolithic black box.  In contrast, side channel attacks circumvent the mathematical complexity of reversing encryption by getting additional data about the encryption process.  [insert diagrams of info. leakage here]  Depending on the algorithm and its implementation, a wide variety of information leaks may exist on a number of different physical channels.
\subsubsection{Timing attacks}
\subsubsection{Caching attacks}
\subsubsection{Branch Predictor attacks}
\subsubsection{Power attacks}
\section{PRET}


\section{Implementation}

\section{Results}


\section{Conclusion}

\bibliographystyle{latex8}
\bibliography{RTencryption}

%-------------------------------------------------------------------------



\end{document}

